% \iffalse meta-comment
%
% File: siunitx-angle.dtx Copyright (C) 2016-2019,2021,2022 Joseph Wright
%
% It may be distributed and/or modified under the conditions of the
% LaTeX Project Public License (LPPL), either version 1.3c of this
% license or (at your option) any later version.  The latest version
% of this license is in the file
%
%    https://www.latex-project.org/lppl.txt
%
% This file is part of the "siunitx bundle" (The Work in LPPL)
% and all files in that bundle must be distributed together.
%
% The released version of this bundle is available from CTAN.
%
% -----------------------------------------------------------------------
%
% The development version of the bundle can be found at
%
%    https://github.com/josephwright/siunitx
%
% for those people who are interested.
%
% -----------------------------------------------------------------------
%
%<*driver>
\documentclass{l3doc}
\ProvideDocumentCommand\foreign{m}{\textit{#1}}
% The next line is needed so that \GetFileInfo will be able to pick up
% version data
\usepackage{siunitx}
\begin{document}
  \DocInput{\jobname.dtx}
\end{document}
%</driver>
% \fi
%
% \GetFileInfo{siunitx.sty}
%
% \title{^^A
%   \pkg{siunitx-angle} -- Formatting angles^^A
%   \thanks{This file describes \fileversion,
%     last revised \filedate.}^^A
% }
%
% \author{^^A
%  Joseph Wright^^A
%  \thanks{^^A
%    E-mail:
%    \href{mailto:joseph.wright@morningstar2.co.uk}
%      {joseph.wright@morningstar2.co.uk}^^A
%   }^^A
% }
%
% \date{Released \filedate}
%
% \maketitle
%
% \begin{documentation}
%
% \section{Formatting angles}
%
% \begin{function}{\siunitx_angle:n, \siunitx_angle:nnn}
%   \begin{syntax}
%     \cs{siunitx_angle:n} \Arg{angle}
%     \cs{siunitx_angle:nnn} \Arg{degrees} \Arg{minutes} \Arg{seconds}
%   \end{syntax}
%   Typeset the \meta{angle} (which may be given as separate \meta{degree},
%   \meta{minute} and \meta{second} components). The \meta{angle} (or components)
%   may be given as expressions. The \meta{angle} should be a number as understood
%   by \cs{siunitx_format_number:nN}, with no uncertainty, exponent or imaginary
%   part.  The unit symbols for degrees, minutes and seconds are |\degree|,
%   |\arcminute| and |\arcsecond|, respectively
% \end{function}
%
% \subsection{Key--value options}
%
% The options defined by this submodule are available within the \pkg{l3keys}
% |siunitx| tree.
%
% \begin{function}{angle-mode}
%   \begin{syntax}
%     |angle-mode| = \meta{choice}
%   \end{syntax}
%   Selects how angles are formatted: a choice from
%   the options |arc|, |decimal| and |input|. The option |arc| means that angles
%   will always be typeset in arc (degree, minute, second) format, whilst
%   |decimal| means that angles are typeset as a single decimal value. The
%   |input| setting means that the input format (\foreign{i.e.}~difference
%   between \cs{siunitx_angle:n} and \cs{siunitx_angle:nnn}) is maintained. The
%   standard setting is |input|.
% \end{function}
%
% \begin{function}
%   {
%     angle-symbol-degree ,
%     angle-symbol-minute ,
%     angle-symbol-second
%   }
%   \begin{syntax}
%     |angle-symbol-degree| = \meta{symbol}
%   \end{syntax}
%   Sets the symbol used for arc degrees, minutes or seconds, respectively.
% \end{function}
%
% \begin{function}{angle-symbol-over-decimal}
%   \begin{syntax}
%     |angle-symbol-over-decimal| = |true|\verb"|"|false|
%   \end{syntax}
%   Determines if the arc separator is printed over the decimal marker, a
%   format used in astronomy. The standard setting is |false|.
% \end{function}
%
% \begin{function}{arc-separator}
%   \begin{syntax}
%     |arc-separator| = \meta{separator}
%   \end{syntax}
%   Inserted between arc parts (degree, minute and second components).
%   The standard setting is |\,|.
% \end{function}
%
% \begin{function}{fill-angle-degrees}
%   \begin{syntax}
%     |fill-arc-degrees| = |true|\verb"|"|false|
%   \end{syntax}
%   Determines whether a missing degrees part is zero-filled when printing an
%   arc. The standard setting is |false|.
% \end{function}
%
% \begin{function}{fill-angle-minutes}
%   \begin{syntax}
%     |fill-arc-minutes| = |true|\verb"|"|false|
%   \end{syntax}
%   Determines whether a missing minutes part is zero-filled when printing an
%   arc. The standard setting is |false|.
% \end{function}
%
% \begin{function}{fill-angle-seconds}
%   \begin{syntax}
%     |fill-arc-seconds| = |true|\verb"|"|false|
%   \end{syntax}
%   Determines whether a missing seconds part is zero-filled when printing an
%   arc. The standard setting is |false|.
% \end{function}
%
% \begin{function}{number-angle-product}
%   \begin{syntax}
%     |number-angle-product| = \meta{separator}
%   \end{syntax}
%   Inserted between the value of an angle and the unit (degree, minute or
%   second component). The standard setting is |\,|.
% \end{function}
%
% \end{documentation}
%
% \begin{implementation}
%
% \section{\pkg{siunitx-angle} implementation}
%
% Start the \pkg{DocStrip} guards.
%    \begin{macrocode}
%<*package>
%    \end{macrocode}
%
% Identify the internal prefix (\LaTeX3 \pkg{DocStrip} convention): only
% internal material in this \emph{submodule} should be used directly.
%    \begin{macrocode}
%<@@=siunitx_angle>
%    \end{macrocode}
%
%  \begin{variable}{\l_@@_tmp_bool, \l_@@_tmp_dim, \l_@@_tmp_tl}
%    Scratch space.
%    \begin{macrocode}
\bool_new:N \l_@@_tmp_bool
\dim_new:N \l_@@_tmp_dim
\tl_new:N \l_@@_tmp_tl
%    \end{macrocode}
%  \end{variable}
%
%  \begin{variable}
%    {
%      \l_@@_symbol_degree_tl    ,
%      \l_@@_symbol_minute_tl    ,
%      \l_@@_symbol_second_tl    ,
%      \l_@@_force_arc_bool      ,
%      \l_@@_force_decimal_bool  ,
%      \l_@@_astronomy_bool      ,
%      \l_@@_separator_tl        ,
%      \l_@@_fill_degrees_bool   ,
%      \l_@@_fill_minutes_bool   ,
%      \l_@@_fill_seconds_bool   ,
%      \l_@@_product_tl
%    }
%    \begin{macrocode}
\keys_define:nn { siunitx }
  {
    angle-mode .choice: ,
    angle-mode / arc .code:n =
      {
        \bool_set_true:N \l_@@_force_arc_bool
        \bool_set_false:N \l_@@_force_decimal_bool
      } ,
    angle-mode / decimal .code:n =
      {
        \bool_set_false:N \l_@@_force_arc_bool
        \bool_set_true:N \l_@@_force_decimal_bool
      } ,
    angle-mode / input .code:n =
      {
        \bool_set_false:N \l_@@_force_arc_bool
        \bool_set_false:N \l_@@_force_decimal_bool
      } ,
    angle-symbol-degree .tl_set:N =
      \l_@@_symbol_degree_tl ,
    angle-symbol-minute .tl_set:N =
      \l_@@_symbol_minute_tl ,
    angle-symbol-second .tl_set:N =
      \l_@@_symbol_second_tl ,
    angle-symbol-over-decimal .bool_set:N =
      \l_@@_astronomy_bool ,
    angle-separator .tl_set:N =
      \l_@@_separator_tl ,
    fill-angle-degrees .bool_set:N =
      \l_@@_fill_degrees_bool ,
    fill-angle-minutes .bool_set:N =
      \l_@@_fill_minutes_bool ,
    fill-angle-seconds .bool_set:N =
      \l_@@_fill_seconds_bool ,
    number-angle-product .tl_set:N =
      \l_@@_product_tl
  }
\bool_new:N \l_@@_force_arc_bool
\bool_new:N \l_@@_force_decimal_bool
%    \end{macrocode}
% \end{variable}
%
% \begin{macro}{\siunitx_angle:n}
% \begin{macro}{\siunitx_angle:nnn}
% \begin{macro}{\@@_arc_convert:n}
%   The first step here is to force format conversion if required. Going to
%   a decimal is easy, going to arc format is a bit more painful: avoid
%   repeating calculations mainly for code readability.
%    \begin{macrocode}
\cs_new_protected:Npn \siunitx_angle:n #1
  {
    \bool_if:NTF \l_@@_force_arc_bool
      { \exp_args:Ne \@@_arc_convert:n { \fp_eval:n {#1} } }
      {
        \siunitx_number_parse:nN {#1} \l_@@_degrees_tl
        \siunitx_number_process:NN \l_@@_degrees_tl \l_@@_degrees_tl
        \tl_set:Nx \l_@@_degrees_tl
          { \siunitx_number_output:NN \l_@@_degrees_tl \q_nil }
        \@@_arc_print:VVV
          \l_@@_degrees_tl
          \c_empty_tl
          \c_empty_tl
      }
  }
\cs_new_protected:Npn \siunitx_angle:nnn #1#2#3
  {
    \bool_if:NTF \l_@@_force_decimal_bool
      {
        \exp_args:Ne \siunitx_angle:n
          { \fp_eval:n { #1 + (#2) / 60 + (#3) / 3600 } }
      }
      { \@@_arc_sign:nnn {#1} {#2} {#3} }
  }
\cs_new_protected:Npn \@@_arc_convert:n #1
  {
    \use:x
      {
        \siunitx_angle:nnn
          { \fp_eval:n { trunc(#1,0) } }
          { \fp_eval:n { trunc((#1 - trunc(#1,0)) * 60,0) } }
          {
            \fp_eval:n
              {
                (
                          (#1 - trunc(#1,0)) * 60
                  - trunc((#1 - trunc(#1,0)) * 60,0)
                )
                * 60
              }
          }
      }
  }
%    \end{macrocode}
% \end{macro}
% \end{macro}
% \end{macro}
%
% \begin{variable}{\l_@@_degrees_tl, \l_@@_minutes_tl, \l_@@_seconds_tl}
%   Space for formatting parsed numbers.
%    \begin{macrocode}
\tl_new:N \l_@@_degrees_tl
\tl_new:N \l_@@_minutes_tl
\tl_new:N \l_@@_seconds_tl
%    \end{macrocode}
% \end{variable}
%
% \begin{variable}{\l_@@_sign_tl}
%   For the \enquote{sign shuffle}.
%    \begin{macrocode}
\tl_new:N \l_@@_sign_tl
%    \end{macrocode}
% \end{variable}
%
% \begin{macro}{\@@_arc_sign:nnn}
% \begin{macro}{\@@_arc_sign:nn}
% \begin{macro}{\@@_extract_sign:nnnnnnnn}
% \begin{macro}[EXP]{\@@_sign:nnnnnnn}
%   To get the sign in the right place whilst dealing with zero filling
%   means doing some shuffling. That means doing processing of each number
%   manually.
%    \begin{macrocode}
\cs_new_protected:Npn \@@_arc_sign:nnn #1#2#3
  {
    \group_begin:
      \keys_set:nn { siunitx }
        {
          input-close-uncertainty = ,
          input-exponent-markers  = ,
          input-open-uncertainty  = ,
          input-uncertainty-signs =
        }
      \tl_clear:N \l_@@_sign_tl
      \@@_arc_sign:nn {#1} { degrees }
      \@@_arc_sign:nn {#2} { minutes }
      \@@_arc_sign:nn {#3} { seconds }
      \tl_if_empty:NF \l_@@_sign_tl
        {
          \clist_map_inline:nn { degrees , minutes , seconds }
            {
              \tl_if_empty:cF { l_@@_ ##1 _tl }
                {
                  \tl_set:cx { l_@@_ ##1 _tl }
                    {
                       { }
                       { \exp_not:V \l_@@_sign_tl }
                       \exp_after:wN \exp_after:wN \exp_after:wN
                         \@@_sign:nnnnnnn
                           \cs:w l_@@_ ##1 _tl \cs_end:
                    }
                  \clist_map_break:
                }
            }
        }
      \clist_map_inline:nn { degrees , minutes , seconds }
        {
          \tl_if_empty:cF { l_@@_ ##1 _tl }
            {
              \exp_args:Ncc \siunitx_number_process:NN
                { l_@@_ ##1 _tl } { l_@@_ ##1 _tl }
              \tl_set:cx { l_@@_ ##1 _tl }
                {
                  \exp_args:Nc \siunitx_number_output:NN
                    { l_@@_ ##1 _tl } \q_nil
                }
            }
        }
      \@@_arc_print:VVV
        \l_@@_degrees_tl
        \l_@@_minutes_tl
        \l_@@_seconds_tl
    \group_end:
  }
\cs_new_protected:Npn \@@_arc_sign:nn #1#2
  {
    \tl_if_blank:nTF {#1}
      {
        \bool_if:cTF { l_@@_fill_ #2 _bool }
          {
            \tl_set:cn { l_@@_ #2 _tl }
              { { } { } { 0 } { } { } { } { 0 } }
          }
          { \tl_clear:c { l_@@_ #2 _tl } }
      }
      {
        \siunitx_number_parse:nN {#1} \l_@@_tmp_tl
        \exp_after:wN \@@_extract_sign:nnnnnnnn \l_@@_tmp_tl {#2}
      }
  }
\cs_new_protected:Npn \@@_extract_sign:nnnnnnnn #1#2#3#4#5#6#7#8
  {
    \tl_if_blank:nTF {#2}
      { \tl_set_eq:cN { l_@@_ #8 _tl } \l_@@_tmp_tl }
      {
        \tl_set:cn { l_@@_ #8 _tl }
          { {#1} { } {#3} {#4} {#5} {#6} {#7} }
        \tl_set:Nn \l_@@_sign_tl {#2}
        \keys_set:nn { siunitx }
          { input-comparators = , input-signs = }
      }
  }
\cs_new:Npn \@@_sign:nnnnnnn #1#2#3#4#5#6#7
  { \exp_not:n { {#3} {#4} {#5} {#6} {#7} } }
%    \end{macrocode}
% \end{macro}
% \end{macro}
% \end{macro}
% \end{macro}
%
%  \begin{variable}{\l_@@_marker_box, \l_@@_unit_box}
%    For \enquote{astronomy style} angles.
%    \begin{macrocode}
\box_new:N \l_@@_marker_box
\box_new:N \l_@@_unit_box
%    \end{macrocode}
%  \end{variable}
%
% \begin{macro}
%   {
%     \@@_arc_print:nnn, \@@_arc_print:VVV,
%     \@@_arc_print_auxi:nnn, \@@_arc_print_auxi:nVn
%   }
% \begin{macro}{\@@_arc_print_auxii:w}
% \begin{macro}{\@@_arc_print_auxiii:n}
% \begin{macro}{\@@_arc_print_auxiv:NN}
% \begin{macro}{\@@_arc_print_auxv:w}
% \begin{macro}{\@@_arc_print_auxvi:n}
%   The final stage of printing an angle is to put together the three parts:
%   this works even for decimal angles as they will blank arguments for the
%   other two parts The need to handle astronomy-style formatting means that
%   the number has to be decomposed into parts.
%    \begin{macrocode}
\cs_new_protected:Npn \@@_arc_print:nnn #1#2#3
  {
    \@@_arc_print_auxi:nVn {#1}
      \l_@@_symbol_degree_tl {#2#3}
    \@@_arc_print_auxi:nVn {#2}
      \l_@@_symbol_minute_tl {#3}
    \@@_arc_print_auxi:nVn {#3}
      \l_@@_symbol_second_tl { }
  }
\cs_generate_variant:Nn \@@_arc_print:nnn { VVV }
\cs_new_protected:Npn \@@_arc_print_auxi:nnn #1#2#3
  {
    \tl_if_blank:nF {#1}
      {
        \bool_if:NTF \l_@@_astronomy_bool
          { \@@_arc_print_auxii:nw {#2} #1 \q_stop }
          {
            \@@_arc_print_auxv:w #1 \q_stop
            \@@_arc_print_auxvi:n {#2}
          }
        \tl_if_blank:nF {#3}
          {
            \nobreak
            \l_@@_separator_tl
          }
      }
  }
\cs_generate_variant:Nn \@@_arc_print_auxi:nnn { nV }
%    \end{macrocode}
%    To align the two parts of the astronomy-style marker, we need to allow
%    for the |\scriptspace|.
%    \begin{macrocode}
\cs_new_protected:Npn \@@_arc_print_auxii:nw
  #1#2 \q_nil #3 \q_nil #4 \q_nil #5 \q_nil #6 \q_nil #7 \q_nil #8 \q_stop
  {
    \mode_if_math:TF
      { \bool_set_true:N \l_@@_tmp_bool }
      { \bool_set_false:N \l_@@_tmp_bool }
    \siunitx_print_number:n {#2#3#4}
    \tl_if_blank:nTF {#6}
      { \@@_arc_print_auxvi:n {#1} }
      {
        \hbox_set:Nn \l_@@_marker_box
          {
            \@@_arc_print_auxiii:n
               { \siunitx_print_number:n {#5} }
          }
        \hbox_set:Nn \l_@@_unit_box
          {
            \@@_arc_print_auxiii:n
              {
                \siunitx_unit_format:nN {#1} \l_@@_tmp_tl
                \siunitx_print_unit:V \l_@@_tmp_tl
                \skip_horizontal:n { -\scriptspace }
              }
          }
        \dim_compare:nNnTF { \box_wd:N \l_@@_marker_box } >
          { \box_wd:N \l_@@_unit_box }
          {
            \@@_arc_print_auxiv:NN
              \l_@@_marker_box
              \l_@@_unit_box
          }
          {
            \@@_arc_print_auxiv:NN
              \l_@@_unit_box
              \l_@@_marker_box
          }
        \hbox_set_to_wd:Nnn \l_@@_marker_box
          \l_@@_tmp_dim
          {
            \hbox_overlap_right:n
              { \box_use_drop:N \l_@@_marker_box }
            \hbox_overlap_right:n
              { \box_use_drop:N \l_@@_unit_box }
            \tex_hfil:D
          }
        \box_use:N \l_@@_marker_box
        \skip_horizontal:N \scriptspace
        \siunitx_print_number:n {#6}
      }
  }
\cs_new_protected:Npn \@@_arc_print_auxiii:n #1
  {
    \bool_if:NTF \l_@@_tmp_bool
      { \ensuremath }
      { \use:n }
        {#1}
  }
\cs_new_protected:Npn \@@_arc_print_auxiv:NN #1#2
  {
    \dim_set:Nn \l_@@_tmp_dim { \box_wd:N #1 }
    \hbox_set_to_wd:Nnn #2
      \l_@@_tmp_dim
      {
        \tex_hss:D
        \hbox_unpack_drop:N #2
        \tex_hss:D
      }
  }
\cs_new_protected:Npn \@@_arc_print_auxv:w
  #1 \q_nil #2 \q_nil #3 \q_nil #4 \q_nil #5 \q_nil #6 \q_stop
  { \siunitx_print_number:n {#1#2#3#4#5} }
\cs_new_protected:Npn \@@_arc_print_auxvi:n #1
  {
    \nobreak
    \l_@@_product_tl
    \siunitx_unit_format:nN {#1} \l_@@_tmp_tl
    \siunitx_print_unit:V \l_@@_tmp_tl
  }
%    \end{macrocode}
% \end{macro}
% \end{macro}
% \end{macro}
% \end{macro}
% \end{macro}
% \end{macro}
%
%    \begin{macrocode}
\keys_set:nn { siunitx }
  {
    angle-mode                = input      ,
    angle-symbol-degree       = \degree    ,
    angle-symbol-minute       = \arcminute ,
    angle-symbol-over-decimal = false      ,
    angle-symbol-second       = \arcsecond ,
    angle-separator           =            ,
    fill-angle-degrees        = false      ,
    fill-angle-minutes        = false      ,
    fill-angle-seconds        = false      ,
    number-angle-product      =
  }
%    \end{macrocode}
%
%    \begin{macrocode}
%</package>
%    \end{macrocode}
%
% \end{implementation}
%
% \PrintIndex
